\section{时间演化}

\subsection{时间演化算符}

量子态$\alpha$,初始时刻$t_0$,记作$\left| \alpha, t_0 \right\rangle$,由此演化到$t$时刻($t > t_0$)的量子态记作$\left| \alpha, t_0; t \right\rangle$。

定义演化算符$U(t, t_0)$,把初始态和由$t_0$演化到$t$的态联系起来,

\begin{equation}
\left| \alpha, t_0; t \right\rangle = U(t, t_0) \left| \alpha, t_0 \right\rangle 
\end{equation}

这样的算符就叫时间演化算符。

由时空的物理直觉出发,我们规定$U$应当满足的性质:

(1)幺正性:初始时刻态矢量归一,则$t$时刻态矢量也要归一,即:

\begin{equation}
\left\langle \alpha, t_0 | \alpha, t_0 \right\rangle = \left\langle \alpha, t_0; t | \alpha, t_0; t \right\rangle = 1
\end{equation}

这意味着:

\begin{equation}
U^\dagger (t, t_0) U (t, t_0) = 1
\end{equation}

(2)合成性质:

\begin{equation}
U(t_2, t_0) = U(t_2, t_1) U (t_1, t_0)
\end{equation}

物理系统由时间$t_0$演化到$t_2$,必须要先经历$t_0$和$t_2$之间的$t_1$。

(3)连续性:对无穷小时间演化$U(t_0+ dt, t_0)$,

\begin{equation}
\lim\limits_{dt \to 0} U(t_0 + dt, t_0) = 1
\end{equation}

我们可以验证,只要我们取无穷小时间演化算符\index{Infinitesimal time-evolution operator:无穷小时间演化算符}为如下形式,

\begin{equation}
U(t_0+dt, t_0) = 1- i \Omega dt
\end{equation}

其中$\Omega$是厄米算符,$\Omega^\dagger = \Omega $,以上性质即可自动得到满足。

由于$\Omega$是厄米算符,$\Omega$可对应一力学量,其量纲为$\frac{1}{[Time]}$,即频率的量纲。类似前面在无穷小平移算符中所讨论过的,我们把$\Omega$改写为$\frac{H }{\hbar}$,$\hbar$是作用量(action)量纲$[Energy] [Time]$,那么$H$就是能量量纲,即哈密顿算符。这样,

\begin{equation}
U(dt ) = 1- \frac{i H dt}{\hbar}
\end{equation}

\subsection{薛定谔方程}

由无穷小演化算符出发,我们可以推出薛定谔运动方程\index{Schr\"{o}dinger equation:薛定谔方程}(S.E.),首先由演化算符的合成性质:

\begin{equation*}
U(t+dt, t_0) = U(t+ dt , t) U(t, t_0) = \left( 1- \frac{i H dt}{\hbar}  \right) U(t, t_0)
\end{equation*}

调整一下等式左右两边的项:

\begin{equation*}
U(t+dt, t_0) - U(t, t_0) = -  i\frac{H}{\hbar} dt U(t, t_0)
\end{equation*}

就可得到演化算符$U$所满足的微分方程:

\begin{equation}
i \hbar \frac{\partial U(t, t_0) }{\partial t} = H U(t, t_0)
\end{equation}

右乘初始时刻态矢量$\left| \alpha, t_0 \right\rangle$,

\begin{equation}
i \hbar \frac{\partial }{\partial t} \left| \alpha, t_0; t \right\rangle  = H \left| \alpha, t_0; t \right\rangle
\end{equation}

这就是通常所说的薛定谔方程。

\subsection*{参考}

J. J. Sakurai, Modern Quantum Mechanics, \S 2.1