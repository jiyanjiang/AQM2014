\section{相互作用绘景}

\subsection{相互作用绘景}

考虑含时的哈密顿,形式上分为两部分,和时间有关的部分归于相互作用$V(t)$

\begin{equation}
H = H_0 + V(t)
\end{equation}

假设$H_0$的本征值问题已知,

\begin{equation}
H_0 \left| n \right\rangle = E_n \left| n \right\rangle
\end{equation}

在$H_0$表象下,假设$t = 0$,系统处于$\left| i \right\rangle$的态,由于$\left| i \right\rangle$是$H_0$的本征态,而非$H$的本征态,并且$V(t) \neq 0$,我们研究的不再是定态(stationary state),随着时间的演化,那些$n \neq i$的量子态也可能被占据。

假设量子态$\left| \alpha \right\rangle$,用$H_0$表象展开:

\begin{equation}
\left| \alpha \right\rangle = \sum\limits_n  \left| n \right\rangle \left\langle n | \alpha \right\rangle
\end{equation}

$t = 0$时,

\begin{equation}
\left| \alpha \right\rangle = \sum\limits_n c_n(0) \left| n \right\rangle
\end{equation}

$t > 0$时,

\begin{equation}
\left| \alpha, t \right\rangle = \sum\limits_n c_n(t) e^{- i E_n t / \hbar} \left| n \right\rangle
\end{equation}

这里$e^{- i E_n t / \hbar}$是$H_0$导致的,我们能否得到一种形式使$c_n (t)$只由相互作用$V(t)$决定。

%\subsection{相互作用绘景}

考虑变换:

\begin{eqnarray}
\left| \alpha, t \right\rangle_I & = & e^{i H_0 t /\hbar}  \left| \alpha, t \right\rangle_S\\
V_I(t) & = & e^{i H_0 t /\hbar} V_S(t) e^{ - i H_0 t /\hbar}
\end{eqnarray}

这里脚标$I$表示相互作用绘景\index{Interaction Picture:相互作用绘景},$S$表示薛定谔绘景。

对某力学量A,

\begin{equation}
\left\langle A \right\rangle = {}_S\left\langle \alpha,t \right| e^{ - i H_0 t /\hbar} e^{i H_0 t /\hbar} A_S e^{ - i H_0 t /\hbar}   e^{i H_0 t /\hbar} \left| \alpha, t \right\rangle_S
\end{equation}

即:

\begin{equation}
\left\langle A \right\rangle = {}_S\left\langle \alpha, t \right| A_S \left| \alpha, t  \right\rangle_S = {}_I\left\langle \alpha, t \right| A_I \left| \alpha,t  \right\rangle_I
\end{equation}

容易证明:

\begin{eqnarray}
i \hbar \frac{\partial }{\partial t} \left| \alpha,t  \right\rangle_I & = & V_I(t) \left| \alpha,t  \right\rangle_I \\
i \hbar \frac{d A_I}{dt}& = & [A_I(t) , H_0 ]
\end{eqnarray}

即在相互作用绘景下,态矢量$\left| \alpha,t  \right\rangle_I$随时间的演化将完全由相互作用$V_I(t)$决定,而算符$A_I(t)$随时间的演化将只由$H_0$决定。

\subsection{拉比公式}

考虑

\begin{equation}
i \hbar \frac{\partial }{\partial t} \left| \alpha,t  \right\rangle_I = V_I(t) \left| \alpha,t  \right\rangle_I
\end{equation}

左乘$\left\langle n \right|$,

\begin{equation}
i \hbar \frac{\partial }{\partial t}  \left\langle n  | \alpha,t  \right\rangle_I = \sum\limits_m \left\langle n \right| V_I(t) \left| m \right\rangle \left\langle m | \alpha,t  \right\rangle_I
\end{equation}

这里

\begin{eqnarray*}
\left\langle n \right| V_I(t) \left| m \right\rangle & = & \left\langle n \right| e^{iH_0t / \hbar} V e^{-iH_0t / \hbar} \left| m \right\rangle  \\
 {} & = & \left\langle n \right| V \left| m \right\rangle e^{i(E_n - E_m)t / \hbar} \\
{} & = & V_{nm} e^{i(E_n - E_m)t / \hbar}
\end{eqnarray*}

即

\begin{equation}
i \hbar \frac{\partial }{\partial t} c_n(t) = \sum\limits_m V_{nm} e^{i \omega_{nm} t} c_m(t)
\end{equation}

这里:$E_n - E_m = \hbar \omega_{nm}$。

考虑双态系统:

\begin{equation}
H = H_0 + V(t) = \left( \begin{array}{cc} E_1 & \gamma e^{i \omega t} \\  \gamma e^{- i \omega t} & E_2  \end{array} \right)
\end{equation}

假设$t=0$时,$c_1 (0) = 1$,$c_2 (0) = 0$,即粒子全部位于“1”态。可求出$t$时,粒子位于“2”态的几率是:

\begin{equation}
|c_2(t)|^2 = \frac{\gamma^2 / \hbar^2}{\gamma^2 / \hbar^2 + (\omega - \omega_{21})^2 / 4} \sin^2 \left[ t \sqrt{ \frac{\gamma^2}{\hbar^2} + \frac{(\omega - \omega_{21})^2}{4} }    \right]
\label{Rabi's formula}
\end{equation}

其中:$\omega_{21} = \frac{E_2 - E_1}{\hbar}$。公式[\ref{Rabi's formula}]即所谓拉比公式(Rabi's formula)。

\subsection{戴逊级数}

定义相互作用绘景下的演化算符$U_I(t, t_0)$,

\begin{equation}
\left| \alpha,t  \right\rangle_I = U_I(t, t_0) \left| \alpha,t_0 \right\rangle_I
\end{equation}

$U_I(t, t_0)$遵从如下微分方程:

\begin{equation}
i \hbar \frac{d}{dt } U_I (t, t_0) = V_I (t) U_I(t, t_0)
\end{equation}

假设初始条件:$U_I(t, t_0)|_{t = t_0} = 1$

迭代一次,解出$U_I(t, t_0)$,

\begin{equation}
U_I(t, t_0) = 1 - \frac{i}{\hbar} \int_{t_0}^t dt' V_I(t') U_I (t', t_0) 
\end{equation}

反复迭代,

\begin{eqnarray*}
U_I(t, t_0) & = & 1 - \frac{i}{\hbar}\int_{t_0}^t dt' V_I(t') +  \\
{} & {} & \left(\frac{-i}{\hbar} \right)^2 \int_{t_0}^t dt' \int_{t_0}^{t'} dt'' V_I (t') V_I(t'') + ... +  \\
{} & {} &  \left(\frac{-i}{\hbar} \right)^n \int_{t_0}^t dt' \int_{t_0}^{t'} dt'' ... \int_{t_0}^{t^{(n-1)}} dt^{(n)} V_I(t') V_I (t'')... V_I(t^{(n)}) \\
{} &  {}  & + ... 
\end{eqnarray*}

这就是所谓戴逊级数\index{Dyson Series:戴逊级数}(Dyson series)。

由于不同时刻的$V_I$不一定是对易的,我们必须保证积分号里$V_I(t') V_I (t'')... V_I(t^{(n)}$各个时间的次序满足$t > t' > t'' > ... > t^{(n-1)} > t^{(n)} >... > t_0$。

\subsection*{参考}

J. J. Sakurai, Modern Quantum Mechanics, \S 5.5, 5.6