\section{路径积分}

\subsection{变换幅}

H.P.下,传播子$K(x,t;x_0,t_0)$可写为初始时刻位置算符本征矢$\left| x_0, t_0 \right\rangle$与终了时刻位置算符本征矢$\left| x,t \right\rangle$的内积。$\left| x_0, t_0 \right\rangle$与$\left| x, t \right\rangle$分别都可构成表示量子态的基矢量。传播子\index{Propagator:传播子}的地位就相当于由初始时刻(A)到终了时刻(B)的表象变换,

\begin{equation}
K(xt; x_0,t_0) = \left\langle x,t | x_0, t_0 \right\rangle
\end{equation}

就相当于是AB表象之间的变换矩阵,现在因为位置的取值是连续的,我们可以管$\left\langle x,t | x_0, t_0 \right\rangle$叫变换函数(transformation function),它把不同时刻($t$和$t_0$)的两套基矢联系起来。

$t> t_0$,在$t$与$t_0$之间我们总可插入一个时刻$t_1$,量子态由$t_0$开始,先演化到$t_1$,最后演化到终了时刻$t$,这相当于我们在变换幅$\left\langle x,t | x_0, t_0 \right\rangle$中间插入一个单位算符:

\begin{equation}
1 = \int d^3 x_1 \left| x_1, t_1 \right\rangle \left\langle x_1, t_2 \right|
\end{equation}

这里$\left| x_1, t_1 \right\rangle$是$t_1$时刻,位置算符的本征矢,它的集合也构成一个基矢。现在变换幅表示为:

\begin{equation}
\left\langle x,t | x_0, t_0 \right\rangle = \int d^3 x_1 \left\langle x,t | x_1, t_1 \right\rangle \left\langle x_1 , t_1 | x_0, t_0 \right\rangle
\end{equation}

上式称为变换幅的合成性质(composition property),利用这一性质我们可以继续插入不同时刻的位置算符的本征矢,直至相邻时间间隔$d t \to 0$为止。当$d t \to 0$时,变换幅将有可能表现为简单的形式,循此思路,费曼提出了量子力学的路径积分表示,这是不同于波动力学和矩阵力学的第三种量子力学的表述方式。

\subsection{路径求和}

假设我们在$t_0$到$t$区间里插入$N$个时刻,$t_1, t_2, ..., t_N$,当$N \to \infty$时,每相邻时间间隔$\frac{t - t_0}{N} \to 0$。

现在,$x_0,t_0$到$x,t$的变换幅可表示为:

\begin{equation}
\left\langle x,t | x_0, t_0 \right\rangle = \int d^3 x_N ... \int d^3 x_1 \left\langle x, t | x_N, t_N \right\rangle \left\langle  ... \right\rangle ... \left\langle x_1, t_1 | x_0 , t_0 \right\rangle
\end{equation}

积分就是求和,被求和的项是:

\begin{equation*}
\left\langle x,t | N \right\rangle \left\langle N | N-1 \right\rangle ... \left\langle 2 | 1 \right\rangle \left\langle 1 | 0 \right\rangle
\end{equation*}

共$N+1$个无穷小时间间隔的传播子相乘,如果我们追踪每个时间的节点的话,首先由$t_0$出发,$0 \to 1$是第一个传播子,0点作为初始点是固定的,但1点作为积分中的哑元需遍历全空间,假如追踪某个特定的1点,将继续传播到2,$1 \to 2$,同样2点也需遍历全空间,……,最后由N点回到终了时刻$t$。

从$t_0$到$t$构成了一个互相连接的传播子路径,但这个路径中只有首、尾两点是固定的,中间N个节点可以有多种选择,并且需遍历整个空间。这样总的传播子(或变换幅)$\left\langle x,t | x_0, t_0 \right\rangle$就相当于是首尾固定情况下所有“各种路径”的求和。

\begin{equation}
\left\langle x,t | x_0, t_0 \right\rangle \propto \sum\limits_{All Paths} \left\langle x, t | x_N, t_N \right\rangle \left\langle  ... \right\rangle ... \left\langle x_1, t_1 | x_0 , t_0 \right\rangle   
\end{equation}

量子力学中真正关键的是相位,现在每一个无穷小传播子都带着一个相位,所有这些相位要加起来才对应某条路径的相位,不同路径的相位还会发生“干涉”,正是这个缘由导致了不同于经典物理的(丰富的)量子行为。

经典物理中涉及路径的基本原理有光程取极值原理(光学中的费马原理\index{Fermat's principle:费马原理})和力学中的最小作用量原理\index{Principle of least action:最小作用量原理}(哈密顿原理)。

前者是说光实际走过的路径是使光程取极值的路径,表示成数学形式就是:

\begin{equation}
\delta \int n(x) dx = 0 
\end{equation}

后者是说对经典力学而言,粒子实际走过的路径是使作用量取极值的路径,即:

\begin{equation}
\delta \int_{t_1}^{t_2} L_c (x, \dot x) dt = 0
\end{equation}

由费马原理可解释光的直线传播,反射、折射定律等。由哈密顿原理,我们可求出粒子运动的动力学方程:

\begin{equation}
\frac{d}{dt } \frac{\partial L_c}{\partial \dot x} - \frac{\partial L_c }{\partial x} = 0
\end{equation}

上式其实就是:$m \ddot x =  - \frac{\partial V}{ \partial x}$

既然通过(和路径相关的)作用量取极值的方法可以建立经典力学,那么我们是否可类似地基于路径和作用量的概念建立量子力学呢?这个思路对物理学家来说并不陌生,比如量子力学的对易式就可与经典力学中的泊松括号类比。

尽管如此量子力学和经典力学的结果还是有很大区别的,比如经典力学求出来的是一条确定的轨迹,而量子力学中的“对各种路径”求和,则表明每一条路径(而非只有某个特殊的路径)都会对最终的变换幅有贡献。我们期待当$\hbar \to 0$时,量子力学的结果会无限趋近经典力学的结果。

\begin{equation*}
Q.M. \xrightarrow{\hbar \to 0} C.M.
\end{equation*}

\subsection{费曼的路径积分}

量子力学的路径积分表示是由费曼完成的,费曼研究生的时候在狄拉克的书里读到:传播子$\left\langle x_2, t_2 | x_1, t_1 \right\rangle$就相当于(correspons to)是:

\begin{equation}
\exp \left[ {\frac{i \int_{t_1}^{t_2} L_c(x, \dot x) dt}{\hbar}} \right]
\end{equation}

积分部分是作用量(Action)$S$,其量纲与普朗克常数$\hbar$相同。$e$指数部分的$ \frac{i S(2,1)}{\hbar}$给定了传播子$\left\langle x_2, t_2 | x_1, t_1 \right\rangle$的相位。但传播子的振幅是什么,狄拉克并没有继续讨论。费曼读到这段后,深感困惑,他想弄明白这个“相当于”指的到底是什么,是“等于”,还是“正比于”,还是别的什么关系……经过计算,费曼发现这个“相当于”是“正比于”的意思,并求出了比例因子。

$\hbar$处于分母位置还能解释为什么当$\hbar \to 0$时,量子行为会过渡到经典行为。量子行为对应的是所有路径求和,而经典路径对应的是只有最小作用量的路径会凸显出来。最小作用量的路径就是使$\delta S = 0$的路径,在这个路径的附近还有很多其它路径,这些路径与最小作用量路径的偏离就是$\delta S$,而$\hbar $是趋于0的,但并不是0,在这个条件下,最小作用量附近路径的相位就是$e^{\frac{i \delta S}{ \hbar}}$,都趋于1,即相长干涉,因此经典路径的贡献就会加强凸显出来。

对一般的路径 $ \delta S \neq 0$,这意味着它和它附近路径的相位差是$\frac {\delta S} { \hbar } $,这里$ \delta S \neq 0 $,而$\hbar \to 0 $,$\frac{\delta S}{\hbar} $ 就是个很大的不确定的数,反映在相位上就是任意相位,这样对一般路径而言,它附近的那些路径与它的相位差是任意取值的,因此会互相抵消掉。这样在 $\hbar \to 0$ 时,那些非经典路径的变换幅就只能取0了。

%当然费曼最感困惑的是为什么狄拉克自己没有继续这个思路,进而提出量子力学的路径积分表示。据说他曾经就此问过狄拉克,但狄拉克没有告诉他。

现在我们来计算某个无穷小的传播子:

\begin{equation}
\left\langle x_n, t_n | x_{n-1}, t_{n-1} \right\rangle = \frac{1}{w(\Delta t)} e^{iS(n, n-1) / \hbar}
\end{equation}

这里$\Delta t = t_n - t_{n-1}$,$\frac{1}{w(\Delta t)}$是比例因子,假设它只和$\Delta t$有关。现在来计算$S(n, n-1)$,

\begin{equation}
S(n, n-1) = \int_{t_{n-1}}^{t_n} \left( \frac{m \dot x^2}{2}  - V(x) \right) dt
\end{equation}

由于$\Delta t \to 0$,我们可以把积分进行改写,

\begin{equation}
S(n, n-1) = \left[  \frac{m}{2} \left( \frac{x_n - x_{n-1} }{\Delta t} \right)^2 -V \left( \frac{x_n + x_{n-1}}{2} \right) \right] \Delta t
\end{equation}

假设我们研究的是自由粒子,$V=0$,现在无穷小的传播子变为:

\begin{equation}
\left\langle x_n, t_n | x_{n-1}, t_{n-1} \right\rangle = \frac{1}{w(\Delta t)} e^{ \frac{i m ( x_n - x_{n-1} )^2 }{ 2 \hbar \Delta t} }
\end{equation}

从样子上看,很像一个钟形函数(bell function,或高斯函数),$\Delta t$正比于钟形函数的宽度,当$\Delta t$趋于零时,钟形函数会变得无穷窄,即趋于一个德尔塔函数的样子。

当$\Delta t \to 0$时,即$t_n \to t_{n-1}$时,

\begin{equation}
\lim\limits_{t_n \to t_{n-1}} \left\langle x_n, t_n | x_{n-1}, t_{n-1} \right\rangle = \delta(x_n - x_{n-1}) 
\end{equation}

我们引入新的变量$\xi = x_n - x_{n-1}$,现在的问题就是当$\frac{1}{w(\Delta t)} =?$的时候,

\begin{equation}
\lim\limits_{\Delta t \to 0}  \frac{1}{w (\Delta t)} e^{\frac{i m \xi^2}{2 \hbar \Delta t}} = \delta (\xi) 
\end{equation}

考虑到德尔塔函数的性质,

\begin{equation}
\int \delta(\xi) d \xi =  \lim\limits_{\Delta t \to 0} \int d \xi  \frac{1}{w (\Delta t)} e^{\frac{i m \xi^2}{2 \hbar \Delta t}}   = 1 
\end{equation}

由高斯积分,

\begin{equation}
\int d \xi e^{\frac{i m \xi^2 }{ 2 \hbar \Delta t}  } = \sqrt{ \frac{ 2 \pi i \hbar \Delta t }{m } }
\end{equation}

我们就得到:

\begin{equation}
\frac{1}{w(\Delta t )} = \sqrt{\frac{m }{2 \pi i \hbar \Delta t} }
\end{equation}

最终我们就得到传播子的费曼路径积分\index{Path Integral:路径积分}表示:

\begin{equation}
\left\langle x_N,t_N | x_0, t_0 \right\rangle = \lim\limits_{N \to \infty} \left( \frac{m }{2 \pi i \hbar \Delta t } \right)^{N/2} \int dx_{N-1} ... \int dx_1 \prod\limits_{n=1}^{n = N} e^{\frac{iS(n, n-1)}{\hbar}} 
\end{equation}

我们可示意性地将其改写为:

\begin{equation}
\left\langle x_N,t_N | x_0, t_0 \right\rangle = \int_{x_0}^{x_N} D[x(t)] e^{ \frac{ i }{ \hbar} \int_{t_0}^{t_N} L_c(x, \dot x) dt }
\end{equation}

如此繁复的积分使得路径积分法在求解(非相对论)量子力学问题时没什么优势,但当使用路径积分求解量子场论(或多体)问题时却会变得很强大。

最后,我们还可验证如此求出的传播子$\left\langle x,t | x_0, t_0 \right\rangle$符合薛定谔方程(S.E.),即:

\begin{equation}
i \hbar \frac{\partial }{\partial t } \left\langle x,t | x_0, t_0 \right\rangle = - \frac{\hbar^2}{ 2 m } \frac{\partial^2 }{\partial x^2 } \left\langle x,t | x_0, t_0 \right\rangle + V (x) \left\langle x,t | x_0, t_0 \right\rangle
\end{equation}

\subsection*{参考}

J. J. Sakurai, Modern Quantum Mechanics, \S 2.5

R. P. Feynman, The Development of the Space-Time View of Quantum Electrodynamics.
\url{http://www.nobelprize.org/nobel_prizes/physics/laureates/1965/feynman-lecture.html}