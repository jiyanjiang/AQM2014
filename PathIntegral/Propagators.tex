\section{传播子}

\subsection{绘景回顾}

薛定谔绘景(S.P.)下,态矢量$\left| \alpha, t \right\rangle$含时,力学量不含时,态矢量随时间的演化符合薛定谔方程\index{Schr\"{o}dinger equation:薛定谔方程}。

\begin{equation}
i \hbar \frac{\partial }{\partial t} \left| \alpha, t \right\rangle = H \left| \alpha, t \right\rangle 
\end{equation}

其解可表示为:

\begin{equation}
\left| \alpha, t \right\rangle = U(t) \left| \alpha \right\rangle
\end{equation}

这里:

\begin{equation}
U(t) = e^{-iHt / \hbar}
\end{equation}

力学量$A$的本征值问题,

\begin{equation}
A \left| a' \right\rangle = a' \left| a' \right\rangle 
\end{equation}

定义了基矢(basis)$\left| a' \right\rangle$,$\left| a' \right\rangle$也不含时。

态矢量$\left| \alpha, t \right\rangle$含时,可以用基矢$\left| a' \right\rangle$展开:

\begin{equation}
\left| \alpha, t \right\rangle = \sum\limits_{a'} \left| a' \right\rangle \left\langle a' | \alpha, t \right\rangle 
\end{equation}

这里含时的部分被包含在因子$\left\langle a' | \alpha, t \right\rangle $中。

对海森堡绘景(H.P.)而言,态矢量不含时,比如就取$t=0$时刻时候的态矢量,记作:$\left| \alpha \right\rangle$。力学量含时,

\begin{equation}
A(t) = U^\dagger (t) A U (t)
\end{equation}

符合海森堡运动方程,

\begin{equation}
i \hbar \dot A = [A, H]
\end{equation}

现在来考虑H.P.下力学量$A(t)$的本征值问题:

\begin{equation}
A(t) \left| a', t \right\rangle = a' \left| a', t \right\rangle
\end{equation}

考虑到,

\begin{equation}
U^\dagger A U U^\dagger \left| a' \right\rangle = a' U^\dagger \left| a' \right\rangle
\end{equation}

H.P.下的基矢是:

\begin{equation}
\left| a', t \right\rangle = U^\dagger \left| a' \right\rangle
\end{equation}

看起来和S.P.下态矢量$\left| \alpha , t \right\rangle$随时间的演化$\left| \alpha , t \right\rangle = U \left| \alpha \right\rangle$类似,但方向正好相反。

\subsection{传播子}

在S.P.下,给定量子系统$H = \frac{p^2}{2m} + V$,考虑态矢量由$\left| \alpha , t_0 \right\rangle$演化到$\left| \alpha , t \right\rangle$,

\begin{equation}
\left| \alpha , t \right\rangle = e^{-i H (t - t_0) / \hbar}  \left| \alpha , t_0 \right\rangle
\end{equation}

考虑$[A, H ] = 0$,基矢$\left| a' \right\rangle$同时也是$H$的本征矢,$H \left| a' \right\rangle = E_{a'} \left| a' \right\rangle $,因此:

\begin{equation}
\left| \alpha , t \right\rangle = \sum\limits_{a'} \left| a' \right\rangle \left\langle a' | \alpha, t_0 \right\rangle e^{-i E_{a'} (t - t_0) / \hbar}
\end{equation}

左乘$\left\langle x' \right|$,把上式变到位置表象,

\begin{equation}
\left\langle x' | \alpha, t \right\rangle = \sum\limits_{a'} \left\langle x' | a' \right\rangle \left\langle a' | \alpha, t_0 \right\rangle e^{- i E_{a'} (t - t_0) / \hbar}  
\end{equation}

把因子$\left\langle a' | \alpha, t_0 \right\rangle$表示为位置表象,

\begin{equation}
\left\langle a' | \alpha, t_0 \right\rangle = \int d^3 x \left\langle a' | x' \right\rangle \left\langle x' | \alpha, t_0 \right\rangle 
\end{equation}

为了避免歧义,把$\left\langle x' | \alpha, t \right\rangle$中的$x'$改记为$x''$,

\begin{equation}
\left\langle x'' | \alpha, t \right\rangle = \int d^3 x' \sum\limits_{a'}  \left\langle x'' | a' \right\rangle \left\langle a' | x' \right\rangle e^{- i E_{a'} (t - t_0) / \hbar} \left\langle x' | \alpha, t_0 \right\rangle
\end{equation}

定义积分的核(Kernel)为$K(x'', t; x', t_0)$,

\begin{equation}
K(x'', t; x', t_0) = \sum\limits_{a'}  \left\langle x'' | a' \right\rangle \left\langle a' | x' \right\rangle e^{- i E_{a'} (t - t_0) / \hbar}
\end{equation}

并用波动力学中的波函数对上上式进行改写,

\begin{equation}
\psi_\alpha (x'', t ) = \int d^3 x' K(x'', t; x', t_0) \psi_\alpha (x', t_0 )
\end{equation}

我们称$K$为传播子\index{Propagator:传播子},即描述由时空中$x', t_0$传播到$x'', t$的行为。传播子与初始时刻的波函数$\psi_\alpha (x', t_0 )$无关,它是物理系统($H$,这里就是$V$)本身性质的反映。

只要知道$\psi_\alpha (x', t_0 )$和$K$,就可知道$t > t_0$时刻的波函数$\psi_\alpha (x'', t )$。即是决定的,是因果律的。

\begin{equation*}
i \xrightarrow{K}{} f
\end{equation*}

\subsection{传播子的性质}

(1)对$t > t_0$,传播子$K$符合薛定谔方程(S.E.)

(2)当$t \to t_0$时,传播子是德尔塔函数,

\begin{equation}
\lim\limits_{t \to t_0} K = \lim\limits_{t \to t_0} \sum\limits_{a'} \left\langle x'' | a' \right\rangle \left\langle a' | x' \right\rangle e^{- i E_a' (t - t_0) / \hbar} = \left\langle x'' | x' \right\rangle
\end{equation}

(3)$K$可表示为:

\begin{equation}
K = \left\langle x'' \right| e^{- i H (t - t_0) / \hbar}  \left| x' \right\rangle  
\end{equation}

证明:

\begin{equation*}
K = \sum\limits_{a'} \left\langle x'' \right| e^{- i H (t - t_0) / \hbar} \left| a' \right\rangle \left\langle a' | x' \right\rangle = ...  
\end{equation*}

(4)$K$是格林函数(G.F.)

与电动力学中给定电荷分布求电动势类比,

\begin{equation}
\phi(x) = \int d^3 x' \frac{\rho (x')}{ \left| x - x' \right| }
\end{equation}

这里$\phi(x)$和$\psi_\alpha (x'', t)$对应,$\rho(x')$和$\psi_\alpha (x', t_0)$对应,而$\frac{1}{ \left| x - x' \right| }$和传播子$K$对应,都是积分的核。

$\frac{1}{ \left| x - x' \right| }$是微分方程$\nabla \cdot \nabla \frac{1}{ \left| x - x' \right| }  = \delta^3 (x- x')$的解,类似地$K$是如下方程的解,

\begin{equation}
\left[ i \hbar \frac{\partial }{\partial t} + \frac{\hbar^2}{ 2m} \nabla''^2 - V(x'')  \right] K = i \hbar \delta^3 (x'' - x') \delta (t - t_0)
\end{equation}

即:

\begin{equation}
\left[ i \hbar \frac{\partial }{\partial t} - H \right] K = i \hbar \delta^3 (x'' - x') \delta (t - t_0) 
\end{equation}

边条件是$t < t_0$时,$K = 0$。对应推迟解,或符合因果律的解(时间在前的影响时间在后的,而非相反)。

$K$就是数学上的格林函数\index{Green Function:格林函数}(Green Function,G.F.),其定义为微分算子作用于格林函数等于德尔塔函数,即$\hat L G(x,s) = \delta(x-s)$。

\subsection{态的求和}

考虑$x'' = x'$,$t_0 = 0$,则$K (x', t ; x', 0)$

定义格林函数为:

\begin{equation}
G(t) = \int d^3 x' K(x', t; x', 0) = ... = \sum\limits_{a'} e^{-i E_{a'} t / \hbar}
\end{equation}

如此定义的G.F.,就等于对所有量子态$a'$的求和(sum over states)。

我们可以把$G(t)$与统计力学中的配分函数\index{partition function:配分函数}(partition function,Z)进行比较,

\begin{equation}
Z = \sum\limits_{a'} e^{- \beta E_{a'}}
\end{equation}

这两个式子都是对所有量子态$a'$的求和。

引入虚时变换,

\begin{equation}
\beta = \frac{i t }{\hbar }
\end{equation}

这两个求和在数学上有相同的形式,这意味着我们在量子力学(量子场论)中发展出来的方法可以直接应用于量子统计。

$G(t)$的宗量是时间,为了更方便地考察其中的物理,我们往往把时间宗量$t$变为能量宗量$E$(或频率宗量$\omega$,考虑到$E = \hbar \omega$,这其实是一回事)。

\begin{equation}
\widetilde{G} (E) = - \frac{i}{\hbar } \int_0^{\infty} dt G(t ) e^{i Et / \hbar}
\end{equation}

但这样积分积出来是不定的(存在$e^{i \infty}$的项),为了避免这个困难,我们把$E$拓展到复平面,并考虑变量变换$E \to E + i 0^+$,这样我们就得到,

\begin{equation}
\widetilde{G} (E) = \sum\limits_{a'} \frac{1}{ E - E_{a'}}
\end{equation}

这意味着我们只需研究格林函数$\widetilde{G} (E)$在复$E$平面的解析性质(奇点位置)就可知道给定物理系统的能谱$\{ E_{a'} \}$。

\subsection{海森堡绘景下的传播子}

传播子$K (x'',t; x',t_0) = \left\langle x'' \right| e^{- i H (t - t_0) / \hbar} \left| x' \right\rangle$是在薛定谔绘景(S.P.)下的定义。

在海森堡绘景\index{Heisenberg Picture:海森堡绘景}下,利用$\left| a' , t \right\rangle = U^\dagger \left| a' \right\rangle$

\begin{equation}
K = \left\langle x'' \right| U(t) U^\dagger (t_0 ) \left| x' \right\rangle = \left\langle x'', t | x', t_0 \right\rangle
\end{equation}

传播子$K$即变换幅\index{Transition amplitude:变换幅}(Transition amplitude),$\left\langle x'', t | x', t_0 \right\rangle$的含义是“粒子”由$x', t_0$出发,传播到$x'', t$处的几率幅。

\subsection*{参考}

J. J. Sakurai, Modern Quantum Mechanics, \S 2.2 \S 2.5