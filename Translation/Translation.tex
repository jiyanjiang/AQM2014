\section{平移算符}

\subsection{无穷小平移算符}

我们把无穷小平移\index{infinitesimal translation:无穷小平移}(infinitesimal translation)$T(dx')$定义为:

\begin{equation}
T(dx') \left| x' \right\rangle = \left| x' + dx' \right\rangle 
\end{equation}

效果上就是把位于$x'$附近的量子态平移到$x'+ dx'$位置附近。

根据物理上我们对空间的直觉,我们规定$T$应满足如下性质:

(1)幺正性:

平移不改变粒子数:$\left\langle \alpha | \alpha \right\rangle = \left\langle \alpha \right| T^\dagger T \left| \alpha \right\rangle  $,即要求:

\begin{equation}
T^\dagger (dx') T (dx') = 1
\end{equation}

(2)对连续的平移操作满足:

\begin{equation}
T( dx'' ) T (dx') = T(dx'' + dx')
\end{equation}

(3)逆操作就是反方向平移:

\begin{equation}
T^{-1} (dx') = T(- d x')
\end{equation}

(4)连续性:

\begin{equation}
\lim\limits_{dx' \to 0} T(dx') = 1
\end{equation}

可以证明,假如$T(dx')$取如下形式,

\begin{equation}
T(dx' ) = 1 - i K \cdot dx'
\end{equation}

并且算符$K$是厄米的($K^\dagger = K$),以上四条性质将自动满足。

由于$K$是厄米算符,根据量子力学它可对应某个力学量,其量纲为$\frac{1}{[ Length ]}$。

物理学首先在人的日常经验中被研究,所以最初的单位制是适合于人本身的尺度的,比如长度是米,时间是秒,质量是千克。而量子力学对应的是原子尺寸的物理现象,这样就涉及到一个换算的问题。$K$常被写为$\frac{p}{\hbar}$,这里$\hbar$是普朗克常数,量纲为$[Energy] [Time]$,$\hbar$很小,就是因为日常世界和量子世界尺度的不同引入的换算因子,因为如果我们首先发展的是量子力学我们本可以取$\hbar = 1$的。

现在$p$就对应某个力学量,其量纲为$\frac{[E] [T]}{[L]} = [M] \frac{[L]}{[T]}$,即动量的量纲。这样我们就把动量算符\index{momentum operator:动量算符}定义为无穷小平移算符的产生算符。

\begin{equation}
T(dx') = 1 - i \frac{p \cdot dx'}{ \hbar }
\end{equation}

\subsection{基础对易式}

我们可由无穷小平移算符$T(dx')$的表达式出发推出基础对易式\index{fundamental commutation relation:基础对易式}$[x, p_x] = i \hbar$。

首先计算对易式$[x, T(dx')]$,

\begin{eqnarray*}
x T (dx') \left| x' \right\rangle & = & (x' + dx' ) \left| x'+ dx' \right\rangle \\
T(dx' ) x \left| x' \right\rangle & = & x' T(dx') \left| x' \right\rangle = x' \left| x' + dx' \right\rangle
\end{eqnarray*}

由于$dx' \to 0$,以下计算只需要考虑$dx'$的一阶小量即可。得到,

\begin{equation}
[x, T(dx')] = dx'
\end{equation}

将上式展开:

\begin{equation}
- i x K \cdot dx' + i K \cdot dx' x = dx'
\end{equation}

需要注意的是以上$x$,$K$和$dx'$均是矢量(算符)。

假设$d \vec x'$在第1个方向上,即:$d \vec x' = d \vec x' \hat x_1 $,$\vec K \cdot d \vec x' = K_1 dx'$。

\begin{eqnarray*}
x_1 K_1 - K_1 x_1 & = & i \\
x_2 K_1 - K_1 x_2 & = & 0 \\
x_3 K_1 - K_1 x_3 & = & 0
\end{eqnarray*}

我们还可以取$d \vec x'$在第2个方向或第3个方向上。这样我们就得到:

\begin{equation}
[x_i , K_j ] = i \delta_{ij}
\end{equation}

$i$,$j$分别都可取1,2,3对应三维物理空间。我们还可把上式改写为以下常见的形式:

\begin{equation}
[x_i , p_j ] = i \hbar \delta_{ij}
\end{equation}

即基础对易式\index{fundamental commutation relation:基础对易式}(fundamental commutation relations)。

\subsection{动量算符}

由无穷小平移算符出发,

\begin{eqnarray*}
\left( 1- \frac{i p \Delta x'}{\hbar}  \right) \left| \alpha \right\rangle & = & \int dx' T(\Delta x') \left| x' \right\rangle \left\langle x' | \alpha \right\rangle \\
 {} & = & \int dx' \left| x' + \Delta x' \right\rangle \left\langle x' | \alpha \right\rangle  \\
 {} & = & \int dx' \left| x' \right\rangle \left\langle x- \Delta x' | \alpha \right\rangle \\
 {} & = & \int dx' \left| x' \right\rangle \left( \left\langle x' | \alpha \right\rangle - \Delta x' \frac{\partial }{\partial x'}  \left\langle x' | \alpha \right\rangle \right)
\end{eqnarray*}

即,

\begin{equation}
p \left| \alpha \right\rangle = \int dx' \left| x' \right\rangle \left( \frac{\hbar}{i} \frac{\partial }{\partial x'} \left\langle x' | \alpha \right\rangle  \right)
\end{equation}

左乘$\left\langle x' \right|$,可得:

\begin{equation}
\left\langle x' \right| p \left| \alpha \right\rangle = \frac{\hbar}{i } \frac{\partial }{\partial x'} \left\langle x' | \alpha \right\rangle
\end{equation}

如果把$\left| \alpha \right\rangle$替换为$\left| x'' \right\rangle$,就得到动量算符\index{momentum operator:动量算符}在位置表象下的矩阵表示:

\begin{equation}
\left\langle x' \right| p \left| x'' \right\rangle = \frac{\hbar}{i } \frac{\partial }{\partial x'} \left\langle x' | x'' \right\rangle = \frac{\hbar}{i } \frac{\partial }{\partial x'} \delta(x' - x'')
\end{equation}

\subsection*{参考}

J. J. Sakurai, Modern Quantum Mechanics, \S 1.6
